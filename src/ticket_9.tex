\section{Взаимное исключение в распределённой системе.
Алгоритмы на основе кворума (простое большинство, рушащиеся стены).}

\begin{definition} \textit{Кворум:}
    \begin{itemize}
        \item Семейство подмножеств множества процессов $Q \subset 2^\bP$.
        \item Любые два кворума имеют непустое пересечение:
            \[
                \forall A, B \in Q\colon~ A \cap B \neq \emptyset
            \]
    \end{itemize}
\end{definition}

\begin{examples} Варианты кворумов:
    \begin{itemize}
        \item Централизованный алгоритм как частный случай кворума.
        \item Простое большинство (больше половины процессов) и взвешенное большинство.
        \item Рушащиеся стены.
    \end{itemize}
\end{examples}

\begin{definition} \textit{Кворум <<рушащиеся стены>>}
    \begin{itemize}
        \item Процессы образуют квадратную матрицу (приблизительно).
        \item Кворумом назовем набор процессов, состоящий из некоторого столбца 
            целиком и представителей всех остальных столбцов.
        \item Заметим, что пересечение любых двух таких множеств непусто.
        \item Размер порядка $2\sqrt{n}$.
    \end{itemize}
\end{definition}

\begin{remark}
    При пересечении кворумов потенциально возможен \textit{deadlock}. Решением служит
    \textit{иерархическая блокировка}.
\end{remark}
