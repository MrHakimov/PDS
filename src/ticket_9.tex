\section{Взаимное исключение в распределённой системе. Алгоритмы на основе кворума (простое большинство, рушащиеся стены).}

\begin{definition} \textit{Кворум:}
    \begin{itemize}
        \item Семейство подмножеств множества процессов $Q \subset 2^\bP$
        \item Любые два кворума имеют непустое пересечение:
            \[
                \forall A, B \in Q\colon~ A \cap B \neq \emptyset
            \]
    \end{itemize}
\end{definition}

\begin{examples} Виды кворумов:
    \begin{itemize}
        \item Централизованный алгоритм как частный случай кворума
        \item Простое большинство (больше половины процессов) и взвешенное большинство
        \item Рушащиеся стены
    \end{itemize}
\end{examples}

\begin{definition} \textit{Кворум <<рушащиеся стены>>}
    \begin{itemize}
        \item Процессы образуют квадратную матрицу (приблизительно)
        \item Кворумом назовем набор процессов, состоящий из некоторого столбца целиком и представителей всех остальных столбцов
        \item Заметим, что пересечение любым двух таких множеств непусто, что удовлетворяет определению кворума
    \end{itemize}
\end{definition}

\begin{remark}
    Не все кворумы тривиальны и плохо мастурбируются. Например, <<рушащиеся стены>> имею размер порядка $2\sqrt{n}$
\end{remark}

\begin{remark}
    При пересечении кворумов потенциально возможен deadlock. Решением служит \textit{иерархическая блокировка}
\end{remark}