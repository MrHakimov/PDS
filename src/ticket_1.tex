\section{Формализм. Логические часы Лампорта (свойства и алгоритм)}

Кратко опишем используемые далее обозначения.

\begin{table}[!ht]
    \centering
    \begin{tabular}{|c|c|} \hline
        Обозначение & Объект \\ \hline
        $P, Q, R, \ldots \in \bP$ & Процессы \\ \hline
        $a, b, c, \ldots \in \bE$ & События в процессах $\texttt{proc(e)} \in \bP$
        \\ \hline
        $m \in \bM$ & Сообщения, $\texttt{snd(m), rcv(m)} \in \bE$. \\ \hline
    \end{tabular}
    \caption{Общие обозначения}
\end{table}

\begin{definition}
    Отношение \textit{Произошло-до ($\rightarrow$)} -- минимальный
    \textit{строгий частичный порядок} (транзитивное, антирефлексивное,
    антисимметричное отношение) на $\bE \times \bE$ такой, что
    \begin{itemize}
        \item Если $e, f$ в одном процессе и $e$ идет перед $f$, то $e \rightarrow f$.
        \item Если $m$ -- сообщение, то $\texttt{snd(m)} \rightarrow \texttt{rcv(m)}$.
    \end{itemize}
\end{definition}

\begin{definition}
    \textit{Логические часы}. Определим функцию $C \colon \bE \to N$ так, чтобы
\[
    \forall e, f \in \bE~ e \rightarrow f \Lra C(e) < C(f)
.\]
\end{definition}

\begin{algorithm}(Логические часы Лампорта)
    \begin{itemize}
        \item Каждый процесс хранит счетчик.
        \item Перед посылкой процесс увеличивает счетчик на единицу.
        \item При посылке дополнительно посылается счетчик.
        \item Получатель обновляет свое время следующим образом:
            \[
                C \leftarrow \max(C, C_r) + 1
            .\]
    \end{itemize}
\end{algorithm}

Свойства логических часов Лампорта: 
\begin{itemize}
    \item Время события не уникально.
    \item Являются логическими часами в смысле определения.
\end{itemize}

