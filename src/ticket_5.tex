\section{Взаимное исключение в распределённой системе. Алгоритм Лампорта}

\begin{table}[!ht]
    \centering
    \begin{tabular}{|c|c|} \hline
        Вид запроса & Действие \\ \hline
        $\texttt{request}$ & От запрашивающего процесса ко всем другим\\ \hline
        $\texttt{ok}$ & Подтверждение получения (не даёт права входа в CS) \\ \hline
        $\texttt{release}$ & После выхода из критической секции \\\hline
    \end{tabular}
    \caption{Виды запросов алгоритма Лампорта}
\end{table}

\begin{algorithm}(Алгоритм Лампорта)
\begin{itemize}
    \item Координатор отсутствует, но у процессов есть приоритет.
    \item Сообщения request и release рассылаются всем другим процессам, всего $3n-3$ сообщения на CS.
    \item Используются логические часы Лампорта.
        Для установления порядка "кто раньше".
        Обязательно требуется порядок FIFO на сообщениях.
    \item Все процессы хранят у себя очередь запросов.
    \item В критическую секцию можно войти, если
        \begin{itemize}
            \item Мой запрос первый в очереди, т.е. его время меньше времени 
                остальных запросов (при равенстве времен порядок определяется 
                по приоритету процесса).
            \item Получен ok от всех других процессов, т.е. они знают о вашем запросе.
        \end{itemize}
    \item Если процесс хочет войти в CS, то он посылает всем другим процессам request со
        своими часами. Ждёт от всех ok. Входит в критическую секцию, если можно.
        Иначе ждем release от всех процессов, которые раньше нас в очереди.
\end{itemize}
\end{algorithm}

