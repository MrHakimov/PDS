\section{СAP теорема (концепции, подходы, без доказательства).}

\begin{definition}
    Распределённой системе хранения необходимо
    три свойства \textbf{CAP}:

    \begin{itemize}
        \item \textit{Consistency} --- все клиенты видят одинаковые данные
            (atomic/strong consistency).
        \item \textit{Availability} --- система работает несмотря на сбои узлов
            (запрос к неотказавшему узлу должен получить ответ).
        \item \textit{Partition tolerance} --- система работает несмотря на обрыв
            связи между разными частями системы (partition).
    \end{itemize}
\end{definition}

\begin{theorem}
    Можно иметь только два из трёх свойств \textbf{CAP}.
\end{theorem}

\begin{definition}
    \textit{Gossip protocol} --- принцип построения системы, при котором
    узлы распространяют информацию друг другу по мере возможности.
    В том числе то, что они узнали от других узлов --- слухи.
\end{definition}

\begin{remark}
    \textit{Gossip protocols} могут обеспечить только \textit{eventual consistency}.
    Это означает, что при отсутствии сбоев через какое-то время все узлы системы
    будут знать согласованное состояние системы.
\end{remark}

\begin{examples} \textit{Варианты систем}:
    \begin{itemize}
        \item \textbf{CA}: тривиальные системы, например, с координатором.
        \item \textbf{CP}: Paxos, Raft и другие.
        \item \textbf{AP}: Gossip protocols.
        \item В некоторых случаях системы принято делить на два вида:
            \begin{itemize}
                \item $ACID = \textbf{CA}$ --- Atomicity, Consistency,
                    Isolation, Durability.
                \item $BASE = \textbf{AP}$ --- Basically Available, Soft state,
                    Eventual consistency.
            \end{itemize}
    \end{itemize}
\end{examples}