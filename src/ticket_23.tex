\section{Синхронные системы. Алгоритм для консенсуса в случае отказа заданного числа узлов.}

\textit{ Из 21 билета. Нельзя прийти к консенсусу, если все 4 свойства системы верны. }

\begin{algorithm} (Попробуем победить отказы в случае синхронной системы.)
    \begin{itemize}
    \item Пусть могут отказать f узлов ($0 \leq f \le N$) Если отказывают все, то кому работать.
    \item Делаем $f + 1$ фазу базового алгоритма (рассылаем известные множества предложений, где одна фаза - это максимальное время доставки сообщения).
    \item Первый раз множество состоит только из своего предложения, затем из всех полученных и своего. Затем уже скомбинированная информация (например, первый узел не успел отослать свое предложение и умер, а второй доотправил его третьему и третий комбинирует информацию о предложениях) отправляется еще раз и так $f + 1$ раз.
    \item Доказательство корректности при отказе узла (по Дирихле) следует из определения алгоритма.
    \item За $f + 1$ фазу в одной фазе нет отказов, и значит все живые процессы корректно передадут свои и соседские предложения. Происходит синхронизация, то есть множества предположений совпадут, а значит и дальше они не поменяются.
    \end{itemize}
\end{algorithm}