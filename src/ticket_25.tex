\section{Синхронные системы. Проблема византийских генералов. Невозможность решения при N = 3, f = 1.}

\incfig{byzantine-3-1}

\begin{theorem} (Невозможность решения задачи византийских генералов при $N \leqslant 3f$)
    \begin{itemize}
        \item Докажем теорему для случая $N = 3$, $f = 1$ и обоснованного консенсуса.
        \item Предположим, существует алгоритм, способный решить эту задачу. Возьмем
            четыре узла, соединим их как на рисунке, и запустим, дав на вход
            $n_{0, 1}$, $n_{0, 2}$ нули как предлагаемые значения, и
            единицы $n_{1, 1}$, $n_{1, 2}$.
        \item Рассмотрим узлы $n_{0, 1}$, $n_{0, 2}$. Представим себе, что $n_{1, 1}$
            и $n_{1, 2}$ -- на самом деле один узел, посылающий странные сообщения.
            Тогда $n_{0, 1}$, $n_{0, 2}$ из обоснованности должны прийти к консенсусу 0.
            Аналогично, $n_{1, 1}$, $n_{1, 2}$ придут к консенсусу 1.
        \item Если мы теперь посмотрим на эту же самую систему с точки зрения
            $n_{0, 1}$, $n_{1, 1}$ и $n_{0, 2}$, $n_{1, 2}$ (все аналогично, просто
            мы теперь считаем, что процессы кооперируют по-другому), то получим,
            что в обеих парах консенсус не достигнут.
    \end{itemize}
\end{theorem}

