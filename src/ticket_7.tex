\section{Взаимное исключение в распределённой системе. Алгоритм обедающих философов.}

\begin{definition}
    В частном случае ресурсы -- вилки, процессы -- философы, граф конфликтов -- кольцо
\end{definition}

\begin{theorem}
    В ориентированном графе без циклов всегда есть исток
\end{theorem}
\begin{theorem}
    Если у истока перевернуть все ребра, то граф останется ациклическим
\end{theorem}

\begin{algorithm}(Алгоритм обедающих философов)
\begin{itemize}
    \item Философ владеет вилкой, если ребро в графе конфликтов исходит из его вершины
    \item Философ может принять пищу, если владеет обеими вилками, т.е. он исток
    \item После еды вилки надо отдать (ленивый способ):
        \begin{itemize}
            \item После еды вилки помечаются грязными
            \item Моем вилки и отдаём их по запросу, даже если сами хотим есть
            \item Чистые вилки не отдаём, если сами хотим есть. Ожидаем все вилки, едим, отдаем, если был запрос
        \end{itemize}
\end{itemize}
\end{algorithm}

\begin{algorithm}(Обобщение алгоритма обедающих философов на произвольный граф)
\begin{itemize}
    \item Взаимное исключение эквивалентно полному графу конфликтов (ребро между каждой парой процессов)
    \item При инициализации вилки раздаются в каком-то порядке (например, по порядку id процессов)
\end{itemize}
\end{algorithm}

\begin{remark} (Результат)
    \begin{itemize}
        \item 0 сообщений на повторный заход в критическую секцию
        \item В худшем случае $2n-2$ сообщения
        \item Количество сообщений пропорционально числу желающих попасть в критическую секцию
    \end{itemize}
\end{remark}

