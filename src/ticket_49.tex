\section{MapReduce. Каскады MapReduce задач, Combiner-оптимизация, Map-only задачи.}

Некоторые задачи нельзя решить за одно исполнение MapReduce.

\lstset{
    language=Python,
    backgroundcolor=\color{black!5}, % set backgroundcolor
    basicstyle=\footnotesize,% basic font setting
}

\begin{example}(Для каждого интервала $[1000k, 1000(k + 1))$ найти число слов с
  числом вхождений, лежащим в данном интервале.)
  \begin{itemize}
    \item С помощью MapReduce посчитать количество вхождений для каждого слова.
    \item С помощью MapReduce по результату предыдущего решить поставленную
      задачу:
      \begin{lstlisting}
fun mapper(doc: [(String, Int)] -> [(Int, String)]):
    for word, count in doc:
        yield count // 1000, word

fun reducer(words: [String]): Int:
    return set(words).size()
      \end{lstlisting}
  \end{itemize}
\end{example}

\begin{definition}
    \textit{Каскад} --- ориентированный ациклический граф MapReduce задач,
    принимающих на вход результат выполнения предыдущих задач.
    Порядок вызовов определяется топологической сортировкой.
    Независимые друг от друга задачи можно решать параллельно.
\end{definition}

\begin{algorithm}(Оптимизация: Combiner)
    \begin{itemize}
        \item Результатом \texttt{map} может быть большое число пар, что может быть избыточным.
        \item Перед передачей корзин редьюсеру, маппер может выполнить свертку по данным
            своего узла локально, чтобы сократить объем передаваемой информации.
    \end{itemize}
\end{algorithm}

\begin{remark}
    \textit{Combiner} может не совпадать с \textit{Reducer}.
\end{remark}

\begin{example}(Подсчет среднего значения по каждому ключу)
  \begin{lstlisting}
fun mapper(doc: [(String, Float)]) -> [(String, Float)]:
    for group, value in doc:
        yield group, value

fun combiner(values: [Float]) -> (Float, Int):
    return (sum(values), len(values))

fun reducer(values: [(Float, Int)]) -> Float:
    return sum(values[0]) / sum(values[1])
  \end{lstlisting}
\end{example}

\begin{definition}
  \textit{Map-only задачи} --- задачи MapReduce, в которых не требуется свертка.
\end{definition}

\begin{example}(Поиск по большой коллекции документов)
  \begin{itemize}
    \item Выполним поиск по каждому документу независимо, весь результат
      сопоставим одному фиктивному ключу.
    \item Объединим результаты с каждого маппера, минуя обработку редьюсерами.
  \end{itemize}
\end{example}
