\section{Слабые модели консистентности. Мотивация, монотонное чтение.}

\textit{Первую часть вопроса см. в предыдущем билете.}

\begin{example}(Монотонное чтение)

    Запись может пропадать при повторном чтении (например, если запросы попадают к разным репликам). Поэтому хотим, чтобы если клиент уже один раз получил результат какого-то изменения, то при дальнейших чтениях он будет продолжать получать этот же результат, а не данные до изменения. Есть два подхода к реализации такой слабой модели консистентности:
    \begin{itemize}
        \item \textit{Запоминание номера последней записи:}
        \begin{itemize}
            \item Аналогично чтению собственных записей.
            \item При \textbf{каждом} запросе узнаём $N$ --- номер последней записи в журнале.
            \item Читаем только из реплики, на которую $N$-ая запись уже отреплицировалась.
            \item Помимо монотонного чтения, такая модель обеспечивает чтение собственных записей.
        \end{itemize}

        \item \textit{Чтение из одной реплики:}
        \begin{itemize}
            \item Будем производить все чтения из одной и той же реплики.
            \item Выбираем ближайшую к клиенту группу реплик --- дата-центр.
            \item Внутри дата-центра определяем номер реплики по хешу.
            \item Проблема: если мы меняем количество реплик в дата-центре, то часть клиентов должна будет начать читать с новой реплики, и они могут столкнуться с немонотонным чтением. Решение: Для уменьшения числа таких клиентов используем консистентное хеширование.
            \item Проблема: после переезда клиент будет делать запросы в другой дата-центр и опять возможно немонотонное чтение. В реальной жизни: скорость передвижения клиента значительно меньше скорости передачи данных, поэтому данные успеют отреплицироваться на реплики нового датацентра.
            \item Проблема: реплика, с которой читал клиент упала, и клиент начинает читать из другой реплики, опять может возникнуть немонотонное чтение. Решение: аналогично изменению числа реплик, обеспечиваем монотонное чтение только в отсутствие сбоев.
        \end{itemize}
    \end{itemize}

\end{example}
