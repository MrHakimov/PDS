\section{Взаимное исключение в распределённой системе. Алгоритм Рикарда и Агравалы}

\begin{table}[!ht]
    \centering
    \begin{tabular}{|c|c|} \hline
        Вид запроса & Действие \\ \hline
        $\texttt{request}$ & От запрашивающего процесса ко всем другим\\ \hline
        $\texttt{ok}$ & Разрешение о входе в критическую секцию \\ \hline
    \end{tabular}
    \caption{Виды запросов алгоритма Рикарда и Агравалы}
\end{table}

\begin{algorithm}(Алгоритм Рикарда и Агравалы)
\begin{itemize}
    \item Оптимизация алгоритма Лампорта.
    \item Всего $2n-2$ сообщений.
    \item Если процесс хочет войти в CS, то он шлет request всем остальным процессам.
        Если процесс получивший запрос не хочет войти в CS, либо его номер
        запроса (в часах) больше, то он отсылает разрешение ok.
        Процесс, который входит в CS, хранит в очереди какие ok-ответы он должен
        послать после выхода.
\end{itemize}
\end{algorithm}

\begin{remark}
    В отличие от алгоритма Лампорта, порядок сообщений FIFO не требуется.
\end{remark}
