\section{Взаимное исключение в распределённой системе. Алгоритм Рикарда и Агравалы}

\begin{table}[!ht]
    \centering
    \begin{tabular}{|c|c|} \hline
    Вид запроса & Действие \\ \hline
    $request$ & От запрашивающего ко всем другим узлам\\ \hline
    $ok$ & После выхода из критической секции \\ \hline
    \end{tabular}
    \caption{Виды запросов алгоритма Рикарда и Агравалы}
\end{table}

\begin{algorithm}(Алгоритм Рикарда и Агравалы)
\begin{itemize}
    \item Оптимизация алгоритма Лампорта
    \item Всего $2n-2$ сообщений
    \item Если узел хочет войти в CS, то он шлет request всем узлам.
        Если узел получивший запрос не хочет войти в CS, либо его номерок запроса (в часах) больше, то он отсылает разрешение ok.
        Узел, который входит в CS, хранит в очереди какие ok-ответы он должен послать после выхода.
\end{itemize}
\end{algorithm}

