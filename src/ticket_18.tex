\section{Упорядочение сообщений. Определения, иерархия порядков. Алгоритм для синхронного порядка.}

\begin{definition}
    Порядок называется \textit{синхронным}, если всем сообщениям можно сопоставить время $T(m)$ так, что $T(\texttt{snd}(m)) = T(\texttt{rcv}(m)) = T(m)$ и
    \[
        \forall e,f \in \bE \colon~ e \rightarrow f
        \Lra T(e) < T(f)
    .\]
\end{definition}

\begin{remark}
    Синхронный порядок является самым сильным требованием о порядке.
\end{remark}

\begin{algorithm}(Централизованный алгоритм для синхронного порядка)

    Выберем координатора, который будет осуществлять передачу. Отличие от централизованного алгоритма для причинно-согласованного порядка заключается в том, что координатор дожидается подтвержедния, что сообщение доставлено и только после этого посылает новое. Для корректности алгоритма также достаточно, чтобы каналы до координатора имели FIFO порядок.
\end{algorithm}

\begin{algorithm}(Распределенный алгоритм для синхронного порядка)

    Алгоритм основан на иерархии процессов. Бывают ``большие`` и ``маленькие`` процессы.
    \begin{itemize}
        \item Большой процесс шлет маленькому сообщение с подтвержением получения. Пока подтвержение не получено, процесс пассивен, не обрабатывает приходящие сообщения и не посылает новые.
        \item Маленький процесс шлет большому сообщение только после получения разрешения, то есть подтвержения возможности посылки. Большой процесс пассивен между отправкой подтверждения и получением сообщения.
        \item Пассивный процесс не участвует в пересылке сообщений, поэтому всегда можно выбрать момент передачи сообщения $T(m)$ в его промежутке пассивности.
        \item Не может произойти взаимная блокировка, потому что становится пассивным всегда более большой процесс.
    \end{itemize}

\end{algorithm}

\begin{remark}
    Независимые пары процессов могут общаться независимо.
\end{remark}
