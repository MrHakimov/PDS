\section{Упорядочение сообщений. Определения, иерархия порядков. Алгоритм для синхронного порядка.}

\textit{Первую часть вопроса см. в билете 16.}

\begin{example}{Нарушение синхронного порядка}
    Перекрестные сообщения между двумя процессами.
\end{example}

\begin{algorithm}(Централизованный алгоритм для синхронного порядка)

    \begin{itemize}
        \item Передача сообщений через координатора.
        \item Координатор дожидается подтверждения, что сообщение доставлено и только после этого может послать новое.
        \item Для корректности необходимо, чтобы каналы до координатора имели FIFO порядок сообщений.
\end{algorithm}

\begin{algorithm}(Распределенный алгоритм для синхронного порядка)

    Алгоритм основан на иерархии процессов, они упорядочены по приоритету.
    \begin{itemize}
        \item Пусть у процесса $P$ приоритет больше, чем у процесса $Q$.
        \item $P$ шлёт сообщение $Q$ с подтверждением получения. Пока подтвержение не получено, процесс пассивен.
        \item $Q$ шлет сообщение $P$ только после получения разрешения, то есть подтвержения возможности посылки. $P$ между отправкой подтверждения и получением сообщения.
        \item Пассивный процесс не участвует в обработке и посылке сообщений, поэтому всегда можно выбрать момент передачи сообщения $T(m)$ в его промежутке пассивности.
        \item Не может произойти взаимная блокировка, потому что пассивным всегда становится процесс с бОльшим приоритетом.
        \item Независимые пары процессов могут общаться независимо.
    \end{itemize}

\end{algorithm}
