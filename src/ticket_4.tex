\section{Взаимное исключение в распределенной системе. Централизованный алгоритм.}

\begin{table}[!ht]
    \centering
    \begin{tabular}{|c|c|} \hline
        Обозначение & Объект \\ \hline
        $CS_i$ & Критическая секция с номером \\ \hline
        $\texttt{Enter}(CS_i)$ & Вход в критическую секцию \\ \hline
        $\texttt{Exit}(CS_i)$ & Выход из критической секции \\ \hline
    \end{tabular}
    \caption{Общие обозначения}
\end{table}

\begin{definition}
    \textit{Взаимное исключение}. Основное требование
    \[
        \texttt{Exit}(CS_i) \to \texttt{Enter}(CS_{i+1})
    .\]
\end{definition}

\begin{definition}
    \textit{Требование прогресса:}
    \begin{itemize}
        \item Каждое желание процесса попасть в критическую секцию будет 
            рано или поздно удовлетворено.
        \item Может быть гарантирован тот или иной уровень честности удовлетворения 
            желания процессов о входе в критическую секцию.
    \end{itemize}
\end{definition}

\begin{algorithm}(Централизованный алгоритм)
    \begin{itemize}
        \item Вся работа контролируется выделенным координатором.
        \item Общение происходит по следующему протоколу:
            \begin{table}[!ht]
                \centering
                \begin{tabular}{|c|c|} \hline
                    Вид запроса & Действие \\ \hline
                    $\texttt{request}$ & Запрос разрешения у координатора \\ \hline
                    $\texttt{ok}$ & Одобрение координатором входа в секцию \\ \hline
                    $\texttt{release}$ & Освобождение пользователем критической 
                    секции \\ \hline
                \end{tabular}
                \caption{Виды запросов}
            \end{table}
        \item При входе в критическую секцию процесс шлёт запрос координатору,
            дожидается разрешения, затем входит в критическую секцию.
            При завершении работы процесс посылает координатору сообщения,
            что секция свободна. Данный алгоритм всегда требует 3 сообщения для
            работы с критической секцией.
        \item Не масштабируется из-за необходимости иметь выделенного координатора.
    \end{itemize}
\end{algorithm}

