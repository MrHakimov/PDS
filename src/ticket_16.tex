\section{Упорядочение сообщений. Определения, иерархия порядков. Алгоритм для FIFO.}

\begin{definition} Иерархия порядков сообщений
    Для обычных сообщений между парой процессов:
    \begin{itemize}
        \item Асинхронная передача (нет порядка).
        \item FIFO.
        \item Причинно-согласованный порядок.
        \item Синхронный порядок.
    \end{itemize}

    Для массовой рассылки сообщений:
    \begin{itemize}
        \item Общий порядок (total order).
    \end{itemize}
\end{definition}

\begin{definition}
    Говорят, что соблюдается порядок \textit{FIFO (First In First Out)}, если
    \[
        \not \exists m,n \in \bM \colon~ \texttt{snd(m)} < \texttt{snd(n)}
        \wedge \texttt{rcv(n)} < \texttt{rcv(m)}
    .\]
\end{definition}

\begin{remark}
    Под $<$ подразумевается отношение порядка в одном процессе.
\end{remark}

\begin{definition}
    Говорят, что соблюдается \textit{причинно-согласованный порядок}, если
    \[
        \not \exists m,n \in \bM \colon~ \texttt{snd(m)} \rightarrow \texttt{snd(n)}
        \wedge \texttt{rcv(n)} \rightarrow \texttt{rcv(m)}
    .\]
\end{definition}

\begin{remark}
    Под $\to$ подразумевается отношение ``произошло до``.
\end{remark}

\begin{definition}
    Порядок называется \textit{синхронным}, если всем сообщениям можно сопоставить время $T(m)$ так, что $T(\texttt{snd}(m)) = T(\texttt{rcv}(m)) = T(m)$ и
    \[
        \forall e,f \in \bE \colon~ e \rightarrow f
        \Lra T(e) < T(f)
    .\]
\end{definition}

\begin{definition}
    Пусть в системе сообщения рассылаются нескольким получателям.
    Обозначим $\texttt{rcv}_p(m)$ -- события получения сообщения процессами
    $p \in \bP$. Будем говорить, что соблюдается \textit{общий порядок},
    если
    \[
        \not\exists m,n \in \bM,~ p,q \in \bP\colon~
        \texttt{rcv}_p(m) < \texttt{rcv}_p(n) \wedge
        \texttt{rcv}_q(n) < \texttt{rcv}_q(m)
    .\]
\end{definition}

\begin{remark}
    Для случая, когда сообщения отправляются только одному процессу, это
    \textit{общий порядок} всегда выполняется.
\end{remark}

\begin{algorithm}(FIFO)

    Алгоритм для восстановления FIFO порядка сообщений основан на их нумерации. Рассмотрим взаимодействие двух процессов:
    \begin{itemize}
        \item Нумеруем все сообщения в порядке отправки.
        \item Получатель поддерживает номер ожидаемого сообщения.
        \item Получатель обрабатывает пришедшее сообщение, если его номер совпал с ожидаемым.
        \item Если номер сообщения не совпал с ожидаемым, то сообщение 
            складывается в очередь и обрабатывается, когда его номер становится 
            равным ожидаемому номеру.
    \end{itemize}
\end{algorithm}
