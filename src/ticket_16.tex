\section{Упорядочение сообщений. Определения, иерархия порядков. Алгоритм для FIFO.}

\begin{definition}
    Говорят, что соблюдается порядок \textit{FIFO (First In First Out)}, если
    \[
        \not \exists m,n \in \bM \colon~ \texttt{snd(m)} < \texttt{snd(n)}
        \wedge \texttt{rcv(n)} < \texttt{rcv(m)}
    .\]
\end{definition}

\begin{remark}
    Под $<$ подразумевается отношение порядка в одном процессе.
\end{remark}

\begin{algorithm}(FIFO)

    Алгоритм для восстановления FIFO порядка сообщений основан на их нумерации. Рассмотрим взаимодействие двух процессов:
    \begin{itemize}
        \item Нумеруем все сообщения в порядке отправки.
        \item Получатель поддерживает номер ожидаемого сообщения.
        \item Получатель обрабатывает пришедшее сообщение, если его номер совпал с ожидаемым.
        \item Если номер сообщения не совпал с ожидаемым, то сообщение 
            складывается в очередь и обрабатывается, когда его номер становится 
            равным ожидаемому номеру.
    \end{itemize}
\end{algorithm}
