\section{Синхронные системы. Проблема византийских генералов. Алгоритм для N >= 4, f = 1. Объяснить идею обобщения для f > 1.}

\begin{definition}
    \textit{Проблема византийских генералов} - прийти к консенсусу, штурмовать или не штурмовать крепость, но из N человек, есть f предадетелей.
\end{definition}

\begin{theorem}
    Решение \textit{проблемы византийских генералов} возможно в синхронной системе только если N > 3f.
\end{theorem}

\begin{algorithm}
    \enewline
    \begin{itemize}
        \item Все процессы шлют свои предложения.
        \item Все процессы пересылают всю полученную информацию всем другим процессам.
        \item Если больше 1 генерала-предателя, то дополнительно пересылаем матрицу ответов, куб,
            гиперкуб и так далее (на каждого генерала по размерности).
        \item Теперь у каждого процесса есть матрица информации от каждого процесса.
        \item Для 4х процессов в матрице испорчена одна строка и столбец.
            Так как матрица 3 на 3 (без диагонали) в каждой строчке можно определить истинное
            значение предложение процесса, просто посчитав самое частое значение в строке
            (смотрите картинку в презентации).
        \item То есть три несбойных процесса имеют одни и те же 4 числа и могут прийти к консенсусу.
        \item Предатель не может помешать прийти к консенсусу, но может повлиять на то какое решение будет принято.
    \end{itemize}
\end{algorithm}
