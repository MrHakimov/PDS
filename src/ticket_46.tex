\section{Кеширование. Cache-Aside и Cache-Through архитектуры, слабые модели консистентности.}

Главная мотивация - уменьшение задержки ответов и сглаживание нагрузки.
Нагрузка на базу данных будет распределена более равномерно, потому что популярные объекты будут часто запрашиваться из кеша.
Задержка при чтении из кеша маленькая, поскольку:
\begin{itemize}
    \item Кеш обычно держит все данные в RAM.
    \item Кеш не поддерживает персистентность.
    \item Кеш не поддерживает сложные индексы (почти все кеш - просто хеш-таблица).
\end{itemize}

\begin{definition}
    \textit{Cache-Aside} -- кеш и СУБД не связаны напрямую, клиент делает запросы в оба хранилища. Пример: MySQL + Memcached.
\end{definition}

\begin{algorithm}(Работа с Cache-Aside)
    \begin{itemize}
        \item Синхронные запросы: сначала запрос в кеш, если там нет, то запрос в базу данных, сохранить в кеш.
            Время на запрос при отсутствии значения в кеше: $2 * RTT(cache) + RTT(DB)$.
        \item Асинхронные запросы: запрос отправляется в кеш и СУБД.
            Время на запрос при отсутствии значения в кеше: $RTT(cache) + RTT(DB)$.
            Если значение есть в кеше, то СУБД работает впустую.
        \item Если у клиента произошел сбой между записью в СУБД и кеш, то в кеше останется старое значение.
            Решение - добавим TTL, чтобы вытеснять устаревшие значения.
    \end{itemize}
\end{algorithm}

\begin{definition}
    \textit{Cache-Through} -- клиент обращается к кеш + СУБД как к единому целому.
\end{definition}

\begin{algorithm}(Работа с Cache-Through)
    \begin{itemize}
        \item Запрос сначала попадает в кеш, если ответ есть, то он возвращается клиенту.
        \item Иначе запрос попадает в основное хранилище.
        \item Ответ попадает в кеш, сохраняется в нём, возвращается клиенту.
        \item Время на запрос при отсутствии значения в кеше: $RTT(cache) + RTT(DB)$.
    \end{itemize}
\end{algorithm}

\begin{algorithm}(Кеширование и слабые модели консистентности)
    \begin{itemize}
        \item Можно думать о кешах как о асинхронных репликах, в которых есть все те же проблемы и такие же способы их решения: чтение собственных изменений - читаем данные которые мы можем изменить из СУБД, без использования кеша, монотонное чтение - читаем из одного и того же кеша.
        \item Можно ли записывать в кеш - это снизит задержку на запись (в СУБД будем писать только когда кеш переполнится), но жертвуем Durability (кеш подтвержил клиенту запись, но упал и значения из RAM'а потерялись).
        \item Если кешей, поддерживающих запись, несколько - будем думать о таких системах как о Multi-Leader Replication.
\end{algorithm}

\begin{algorithm}(Определение несуществования объекта в СУБД с помощью кеша)
    \begin{itemize}
        \item Будем кешировать подотрезок, отсортированного отрезка, который хранится в СУБД, по этому подотрезку можно понять отсутствие некоторых элементов в СУБД.
        \item Если запрашиваемый элемент лежит в границах подотрезка, то достаточно проверить наличие.
        \item Иначе запрос напрявляется в основное хранилище.
    \end{itemize}
\end{algorithm}
