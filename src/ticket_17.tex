\section{Упорядочение сообщений. Определения, иерархия порядков. Алгоритм для причинно-согласованного порядка.}

\textit{Первую часть вопроса см. в билете 16.}

\begin{algorithm}(Централизованный алгоритм для причинно-согласованного порядка)

    \begin{itemize}
        \item Передача сообщений через координатора.
        \item Для корректности необходимо, чтобы каналы до координатора имели FIFO порядок сообщений.
    \end{itemize}
\end{algorithm}

\begin{algorithm}(Распределенный алгоритм для причинно-согласованного порядка)

    \begin{itemize}
        \item Используем матричные часы:
        \begin{itemize}
            \item У каждого процесса хранится матрица $M$. $M_{ij}$ -- количество сообщений, посланных от процесса $P_i$ к процессу $P_j$.
            \item Перед посылкой сообщения от $P_i$ к $P_j$ обновляем $M_{ij} = M_{ij} + 1$ и шлем матрицу $M$ вместе с сообщением.
        \end{itemize}
        \item Если процесс $P_i$ получил сообщение от $P_j$ с матрицей $W$, то оно обработается, если соблюдаются условия:
        \begin{itemize}
            \item FIFO порядок: $W_{ji} = M_{ji} + 1$. Сообщение имеет ожидаемый номер.
            \item Причинная согласованность: $\forall k \not = j \colon~ M_{ki} \geqslant W_{ki}$. То есть посылающий процесс не знает о событиях, о которых не знает принимающий.
        \end{itemize}
        \item При невыполнении хотя бы одного из условий сообщение кладется в очередь.
        \item После обработки обновляется матрица принимающего процесса через покомпонентный максимум: $M = \max{(M, W)}$.
    \end{itemize}

\end{algorithm}
