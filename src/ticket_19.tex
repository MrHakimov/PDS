\section{Общий порядок (total order). Алгоритм Лампорта.}

\textit{Первую часть вопроса см. в билете 16.}

\begin{remark}
    \begin{itemize}
        \item \textit{Broadcast} -- всем другим процессам.
        \item \textit{Multicast} -- подмножеству процессов.
    \end{itemize}
\end{remark}

\begin{algorithm}(Централизованный алгоритм обеспечения общего порядка)

    Пусть в системе соблюдается FIFO порядок сообщений. Если процесс
    хочет сделать рассылку сообщения, он сообщает об этом координатору,
    который в свою очередь рассылает сообщения в фиксированном порядке.
\end{algorithm}

\begin{remark}
    Централизованный алгоритм также обеспечивает причинно-согласованный порядок.
\end{remark}

\begin{algorithm}(Лампорт)

    Обобщим алгоритм Лампорта для взаимной блокировки.
    \begin{itemize}
        \item Необходимо соблюдение FIFO порядка сообщений.
        \item Все multicast-сообщения придется заменить на broadcast.
        \item Перед отправкой сообщения процесс берет ``билет'', соответствующий его
            логическому времени, и посылает \texttt{request} запрос всем другим процессам.
        \item Процессы отвечают ему \texttt{ok}. Отправитель начинает рассылку после
            \texttt{ok} от всех процессов.
        \item Порядок обработки сообщений определяется парой из билета и номера процесса.
    \end{itemize}
\end{algorithm}
