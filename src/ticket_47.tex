\section{MapReduce. Последовательная реализация, примеры решаемых задач.}

\begin{definition}
    \textit{Модель MapReduce}. Универсальная модель для распределенных
    вычислений. Решает задачи, которые можно представить через следующие
    операции:
    \begin{itemize}
      \item \texttt{map :: Document -> [(Key, Value)]}, обработка каждого
        документа независимо.
      \item \texttt{group :: [[(Key, Value)]] -> [(Key, Value)]},
        результаты \texttt{map} объединяются по ключу.
      \item \texttt{reduce :: [Value] -> Result}, свертка сгруппированных
        значений.
    \end{itemize}
\end{definition}

\begin{algorithm}(Последовательная реализация)
  \begin{lstlisting}
fun mapReduce(docs: [Document],
              mapper: Document -> [(Key, Value)],
              reducer: [Value] -> Result): {Key: Result}
    kvs = {}
    for doc in docs:
        for key, value in mapper(doc):
            if key not in kvs:
                kvs[key] = []
            kvs[key].append(value)
    return kvs.map { key, values -> key, reducer(values) }
  \end{lstlisting}
\end{algorithm}

\begin{example}(Распределенный подсчет встречаемости слов)
  \begin{itemize}
    \item Каждый документ разбивается на слова.
    \item \texttt{map} для каждого слова возвращает пару из этого слова и $1$.
    \item \texttt{reduce} суммирует значения.
  \end{itemize}
\end{example}

\begin{example}(Распределенный подсчет обратного индекса слов по документам)
  \begin{itemize}
    \item Каждый документ разбивается на слова.
    \item \texttt{map} для каждого слова возвращает пару из этого слова и
      идентификатора документа.
    \item \texttt{reduce} создает по списку идентификаторов множество
      уникальных.
  \end{itemize}
\end{example}
